% Ogolnie o sieciach neuronowych jak sa wykorzystywnae
% Jak sa wykorzystywane w grach i jaka jest ich przyszlosc
% Jakies przyklady gier z sieciami neuronowymi
% czym 
Wraz ze zwiększającą się popularnością sztucznej inteligencji bazowanej na sieciach neuronowych, możemy zaobserwować ich powszechność w praktycznie każdej dziedzinie od informatki po medycynę. Możliwość wytrenowania sieci neuronowych do rozpoznawania pojawiających się wzorców przy użyciu zarówno uporządkowanych, jak i nieuporządkowanych zbiorów danych daje im niesamowitą przewagę nad tradycyjnymi algorytmami, gdzie zaprogramowanie odpowiednich zasad i procedur, dla bardzo ogólnych przypadków, które wymagają umiejętności nauczenia się z programistycznego punktu widzenia abstrakcyjnych konceptów w celu rozpoznawania danego wzorca, byłoby bardzo trudne, jeśli nie niemożliwe. Dodatkowo powstały metody nauki polegające na udostępnieniu  sztucznej inteligencji środowiska, na które może oddziaływać, gdzie wraz z podejmowanymi akcjami sieć jest trenowana bazując na jej działaniach na dane środowisko, które jest oceniane pod kątem rezultatu jaki chcemy uzyskać.

Nie ma dziś dziedziny, która nie byłaby w stanie skorzystać z tej technologii poczynając od oddania w ręce komputera prostych zadań, które dla człowieka są jedynie lekką niedogodnością, przez programy usuwające potrzebę istnienia niektórych profesji, po ogromne algorytmy obserwujące opinie, zachowania, intencje i wybory całych populacji tym samym modyfikując tkaninę, z której zbudowane jest społeczeństwo.


Z biegiem lat tworzenie i trenowanie sieci neuronowych stawało się coraz przystępniejsze z uwagi na coraz to bardziej rosnące środowisko otaczające oprogramowanie związane z sieciami neuronowymi. Publicznie dostępne i darmowe biblioteki takie jak: TensorFlow, Keras, czy PyTorch oraz ich popularność wraz z zatrważającą dostępnością materiałów w internecie sprawiły, że skonstruowanie i wytrenowanie takiej sztucznej inteligencji, nawet dla prostych zadań, które nie wymagają znajomości i zrozumienia w pełni ich działania, stało się bardzo proste, a moc współczesnych komputerów, nawet tych konsumenckich, którego użytkownik zajmuje się głównie rzeczami nie wymagającymi dużych możliwości obliczeniowych, takimi jak praca biurowa czy konsumpcja multimediów, jest w pełni wystarczająca. Pomimo tego proste problemy, które mogą być w ten sposób rozwiązane, często nie doczekują się rozwiązania. Jest to spowodowane postrzeganiem zagadnienia jako posiadające względnie bardzo wysoki próg wejścia.
% no ze umie robic crazy rzeczy jak czlowiek tylko robot
% ze jest boom na to i tearz kazdy to moze robic
% no ale jest troche upeirdliwe to bo trzeba troche wiedziec co sie robi 
% tak wiec dlatego napisalem prosty i go sparametryzowalem i jest dla kazdego bo latwe

% dlaczego ja stusujemy
% jak pomaga
