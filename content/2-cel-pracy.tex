Celem pracy jest stworzenie sieci neuronowej zdolnej do samodzielnej nauki gry w dowolną grę zręcznościową. Model sieci neuronowej będzie oparty o uczenie przez wzmacnianie w połączeniu z Q-learning'iem. Zdecydowano, że sieć zostanie napisana w języku Python, przy pomocy biblioteki PyTorch. W celu dystrybucji sieci neuronowej do użytku, dla różnych gier stworzona zostanie paczka języka Python, przy pomocy narzędzia Cookiecutter, która zostanie opublikowana na oficjalnym indeksie PyPI. W celu przetestowania paczki oraz sieci neuronowej zostanie stworzona gra w stylu Snake, przy użyciu biblioteki PyGame w celu wizualizacji rozgrywki.
\newline

Układ pracy jest następujący. Rozdział 3 przedstawia użyte narzędzia oraz technologie używane przy budowie sieci neuronowej, paczki Python oraz gry w stylu Snake służącej do testów. Rozdział 4 omawia czym są sieci neuronowe, czym jest model uczenia przez wzmacnianie w połączeniu z Q-learningiem oraz jego implementację w projekcie. W rozdziale 5~omówiony zostaje system paczek języka Python oraz jego system indeksowania, narzędzia takie jak Cookiecutter służące do prostego tworzenia paczek z szablonów, następnie proces tworzenia, opublikowania i zainstalowania utworzonej paczki. W 6 rozdziale zaprezentowane zostanie użycie paczki z siecią neuronową na stworzonej, dla przykładu grze typu Snake. W rozdziale 7 zostanie podsumowana cała praca inżynierska. Zostaną tam wyciągnięte wnioski oraz spostrzeżenia powstałe podczas tworzenia projektu oraz przedstawione zostaną możliwości rozwinięcia projektu.

