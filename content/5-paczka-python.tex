\section{PyPI}

\textbf{PyPI}, \cite{PyPI} \cite{AdvancedPythonDevelopment} czyli \textbf{Python Package Index} jest oficjalnym repozytorium, dla języka \textbf{Python} przeznaczonym na third-party paczki przeznaczone, dla języka, które może zainstalować każdy użytkownik. \textbf{PyPI} jest operowany przez \textbf{Python Software Foundation}, która jest non-profit organizacją odpowiedzialną za utrzymywanie języka \textbf{Python} i jego  środowiska. Niektóre menadżery paczek takie jak \textbf{pip} \cite{Pip} używają \textbf{PyPI} jako domyślnego źródła paczek i ich zależności. Na dzień 30 października 2021 roku, ponad $336 \ 000$ \textbf{Python}'owych paczek może zostać w ten sposób zainstalowane. \textbf{PyPI} głównie hostuje \textbf{Python}'owe paczki w formie nazywanej ``\textbf{sdists}'', czyli z \textit{ang. ``\textbf{source distributions}''}, lub w formie pre-kompilowanej tak zwanej ``\textbf{wheels}''. \textbf{PyPI} jako indeks paczek pozwala użytkownikowi na wyszukiwanie paczek za pomocą konkretnych słów lub filtrów, przeszukując metadane paczek, takie jak licencje czy kompatybilność ze standardem \textbf{POSIX}. Pojedyncza paczka na \textbf{PyPI} może posiadać, pomijając metadane, poprzednie wersje paczki, pre-kompilowane \textbf{wheel}'e (na przykład zawierające \textbf{DLL}'e na system Windows), lub innych form przewidzianych na inne systemy operacyjne i wersje \textbf{Python}'a.

\textbf{Python Distribution Utilities}, czyli w skrócie nazywane ``\textbf{distutils}'', to moduł dodany do \textbf{Python}'a, we wrześniu roku 2000, jako część standardowej biblioteki języka w wersji \lcode{1.6.1}, 9 lat po wyjściu pierwszej oficjalnej wersji \textbf{Python}'a, mając na celu uproszczenie instalacji third-party paczek.
Jednakże, \textbf{distutils} dawało jedynie narzędzie służące do tworzenia paczek, dla kodu napisanego w \textbf{Python}'ie, i nic więcej. Umiało zebrać, i dystrybuować metadane paczki, ale nie używało ich w żaden inny sposób. \textbf{Python} ciągle potrzebował scentralizowanego katalogu paczek znajdującego się w internecie. \textbf{PEP 241} \cite{PEP241} zaproponował standaryzacje indeksów, która została sfinalizowana w marcu 2001 roku.


\clearpage
\section{Struktura paczki Python}

\section{Cookiecutter}
\textbf{Cookiecutter} to command-line'owe narzędzie, którego zadaniem jest tworzenie projektów z tak zwanych \textbf{cookiecutters}, czyli gotowych szablonów, za pomocą, których \textbf{Cookiecutter} umie stworzyć szablon paczki.

W tym celu użyty jest system templatowanie \textbf{Jinja2}, która potrafi inteligentnie zastąpić imiona i strukturę folderów oraz plików i ich zawartości. Co bardzo dobrze pokazuje poniżej zamieszczony przykład:

\begin{figure}[h]
    \centering
    \bigimage{img/cookiecutter_example.png}
    \caption{Przykład działania Cookiecutter'a \cite{CookiecutterExample}}
    \label{img:cookiecutter_example}
\end{figure}


\clearpage
\section{Tworzenie mojej paczki}


\begin{onepage}
    \begin{lstlisting}[
    caption={Użytek Cookiecutter'a},
    label=code:cookiecutter_gh,
    language={Bash},
    escapechar=`,
    numbers=none
]
$ cookiecutter gh:giswqs/pypackage
full_name [Qiusheng Wu]: Jan Bronicki
email [admin@example.com]: janbronicki@gmail.com
github_username [giswqs]: John15321
project_name [Python Boilerplate]: Simple Game AI
project_slug [simple_game_ai]: sgai
project_short_description: Simple Game AI package allows to easily define an
interface for a game and train a simple neural net to play it
pypi_username [John15321]: John15321
version [0.0.1]: 0.0.1
use_pytest [n]:
add_pyup_badge [n]:
create_author_file [n]:
Select command_line_interface:
1 - No command-line interface
2 - Click
3 - Argparse
Choose from 1, 2, 3 [1]: 1
Select open_source_license:
1 - MIT license
2 - BSD license
3 - ISC license
4 - Apache Software License 2.0
5 - GNU General Public License v3
6 - Not open source
Choose from 1, 2, 3, 4, 5, 6 [1]: 1
Select github_default_branch:
1 - main
2 - master
Choose from 1, 2 [1]: 2
\end{lstlisting}
\end{onepage}





Za pomocą komendy \lcode{tree} możemy zobczyć jaką strukturę paczki wygenerował, dla nas \textbf{Cookiecutter}:

\begin{onepage}
    \begin{lstlisting}[
    caption={Wygenerowana struktura paczki za pomocą Cookiecutter'a},
    label=code:tree_sgai_cookiecutter_template,
    language={Bash},
    numbers=none
]
$ tree
.
|-- docs
|   |-- contributing.md
|   |-- faq.md
|   |-- index.md
|   |-- installation.md
|   |-- overrides
|   |   `-- main.html
|   |-- sgai.md
|   `-- usage.md
|-- LICENSE
|-- MANIFEST.in
|-- mkdocs.yml
|-- README.md
|-- requirements_dev.txt
|-- requirements.txt
|-- setup.cfg
|-- setup.py
|-- sgai
|   |-- __init__.py
|   `-- sgai.py
`-- tests
    |-- __init__.py
    `-- test_sgai.py
    \end{lstlisting}
\end{onepage}



\begin{onepage}
    \begin{lstlisting}[
    caption={Struktura gotowej paczki},
    label=code:ckrzak,
    language={Bash},
    numbers=none
]
$ tree
.
|-- AUTHORS.rst
|-- docs
|   |-- authors.rst
|   |-- contributing.md
|   |-- faq.md
|   |-- index.md
|   |-- installation.md
|   |-- overrides
|   |   `-- main.html
|   |-- sgai.md
|   `-- usage.md
|-- LICENSE
|-- MANIFEST.in
|-- mkdocs.yml
|-- README.md
|-- requirements_dev.txt
|-- requirements.txt
|-- setup.cfg
|-- setup.py
|-- sgai
|   |-- agent
|   |   |-- agent.py
|   |   |-- config.py
|   |   |-- data_helper.py
|   |   |-- __init__.py
|   |   |-- model.py
|   |   `-- trainer.py
|   |-- __init__.py
|   `-- sgai.py
`-- tests
    |-- __init__.py
    `-- test_sgai.py
    \end{lstlisting}
\end{onepage}





\begin{onepage}
    \begin{lstlisting}[
    caption={Struktura gotowej paczki},
    label=code:ckrzak,
    language={Bash},
    numbers=none
]
$ python setup.py sdist
...
Writing sgai-0.0.3/setup.cfg
creating dist
Creating tar archive
    \end{lstlisting}
\end{onepage}


\begin{onepage}
    \begin{lstlisting}[
    caption={Struktura gotowej paczki},
    label=code:ckrzak,
    language={Bash},
    numbers=none
]
$ tree
.
...
|-- dist
|   `-- sgai-0.0.3.tar.gz
...
    \end{lstlisting}
\end{onepage}


\begin{onepage}
    \begin{lstlisting}[
    caption={Struktura gotowej paczki},
    label=code:ckrzak,
    language={Bash},
    numbers=none
]
$ twine upload ./dist/sgai-0.0.3.tar.gz
Uploading distributions to https://upload.pypi.org/legacy/
Enter your username: John15321
Enter your password:
Uploading sgai-0.0.3.tar.gz

View at:
https://pypi.org/project/sgai/0.0.3/
    \end{lstlisting}
\end{onepage}


\begin{figure}[h]
    \centering
    \bigimage{img/sgai_on_pypi.png}

    \caption{Paczka sgai widoczna na stronie PyPI}
    \label{img:sgai_on_pypi}
\end{figure}




\begin{onepage}
    \begin{lstlisting}[
    caption={Struktura gotowej paczki},
    label=code:ckrzak,
    language={Bash},
    numbers=none
]
$ pip install sgai

Collecting sgai
  Downloading sgai-0.0.3.tar.gz (7.4 kB)
  Preparing metadata (setup.py) ... done
  ...

$ pip list

Package           Version
----------------- -------
...
sgai              0.0.3
...
\end{lstlisting}
\end{onepage}



\begin{onepage}
    \begin{lstlisting}[
    caption={Struktura gotowej paczki},
    label=code:ckrzak,
    language={Bash},
    numbers=none
]
$ ipython

In [1]: import sgai


In [2]: sgai?
Type:        module
String form: <module 'sgai' from '/Users/jbron/sgai/sgai/__init__.py'>
File:        ~/sgai/sgai/__init__.py
Docstring:   Top-level package for Simple Game AI.

\end{lstlisting}
\end{onepage}






\section{Publikowanie paczki}
