\section{PyPI}
\section{Budowa paczki}
\section{Architektura paczki}
\section{Cookiecutter}
\section{Publikowanie paczki}
\subsection{Pip}
\subsection{Twine}
\subsection{Wheels}
\section{Dokumentacja paczki}

\paragraph{Ukryty link do spisu treści}
Kliknij na numer strony, a pod nim masz ukryty odnośnik do spisu treści. Bardzo ułatwia skakanie po pdf'ie.

\paragraph{Używasz vscode?}
Zobacz na polecane wtyczki w README.

\paragraph{Jak wydzilić z tesktu}
Jak cokolwiek chcesz oddzielic z tekstu możesz użyć \lcode{onepage[Xmm]}.

\begin{onepage}[1cm]
    \centering
    Alias dolor odit. Velit et ut harum. Quos ullam enim suscipit qui omnis dolorum.
\end{onepage}

Reszta tekstu.
Doloremque voluptas sit mollitia eos ut aut. Qui et distinctio vitae. Possimus et in. Est provident qui sequi est nobis cupiditate magni. Recusandae animi aut non ea autem ipsam dolores hic repellendus.

I chyba nic więcej nie potrzeba. Powodzenia!
