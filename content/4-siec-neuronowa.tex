\section{Sieć neuronowa}

Sieć neuronowa \cite{NeuralNetwork} \cite{ArtificialNeuralNetwork} \cite{DeepLearningFromScratch} \cite{HandsOnMachineLearning} jest siecią bądź układem neuronów, w nowoczesnym zrozumieniu pojęcie jest stosowane głównie w odniesieniu do sztucznych sieci neuronowych , które składają się ze sztucznych komputerowo symulowanych, uproszczonych modeli neuronów. Sztuczne sieci neuronowe są inspirowane przez biologiczne sieci neuronowe, które tworzą mózgi zwierząt.
Każde połączenie, czyli inaczej każda synapsa w biologicznej sieci neuronowej może transmitować sygnał do innych neuronów. Sztuczny neuron otrzymuje sygnał, następnie przetwarza go i może zasygnalizować inne połączone z nim neurony. Sam sygnał jest liczbą rzeczywistą, wyjście każdego neuronu obliczane za pomocą jakiejś nie liniowej sumy wejścia. Połączenia neuronów posiadają wagę, która jest dopasowywana podczas procesu uczenia. Waga danego połączenia zwiększa lub pomniejsza siłę sygnału w danym połączeniu. Neurony mogą posiadać granice wartości takiego sygnału, taką, że dany sygnał zostanie przepuszczony jeśli zsumowany sygnał przekracza jakąś wartość. Zazwyczaj neurony są agregowane w tak zwane warstwy. Różne warstwy mogą powodować różne transformacje sygnału bazując na ich wejściach. Sygnał podróżuje od warstwy wejściowej, do ostatniej warstwy, która jest warstwą wejściową, zwykle po przejściu kilku wewnętrznych warstw.


\begin{figure}[h]
    \centering
    \smallimage{img/Colored_neural_network.svg.png}
    \caption{Symboliczny model sieci neuronowej \cite{ColoredNeuralNetwork}}
    \label{img:nn_symbolic_diagram}
\end{figure}


\clearpage

\section{Budowa sieci}




\section{Optymizator}



\section{Strategia nauczania}



\section{Dane wejściowe i wyjściowe modelu}






\begin{figure}[h]
    \centering
    \bigimage{img/GameAI_Diagram.png}
    \caption{Symboliczny model sieci w połączeniu z grą}
    \label{img:rf_learning_diagram}
\end{figure}