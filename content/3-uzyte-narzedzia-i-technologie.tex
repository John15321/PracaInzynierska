% \section{Użyte narzędzia i technologie}



\section{Język Python}
Język Python \cite{IntroducingPython} \cite{AdvancedPythonDevelopment} \cite{ExpertPythonProgramming} jest bardzo popularnym, nowoczesnym oraz wysokopoziomowym językiem programowania. Czytając artykuły i inne treści na temat historii Python'a \cite{HistoriaPythona} \cite{WikipediaPythonProgrammingLanauge} dowiadujemy się, że język powstał w 1991 roku. Stworzony przez Guido van Rossum podczas swojej pracy w laboratorium w Centrum Matematyki i Informatyki w Amsterdamie, pierwotnie tworzony był z myślą zastąpienia rozwijanego w latach osiemdziesiątych języka ABC. Samą nazwę język Python nazwę zawdzięcza popularnemu serialowi komediowemu emitowanemu przez BBC w latach siedemdziesiątych - ``Latający Cyrk Monty Pythona'', którego Guido był fanem.
Projekt początkowo zakładał stworzenie prostego w użyciu, zwięzłego i wysokopoziomowego języka, głównie na potrzeby pracy w Amsterdamskim laboratorium. Z biegiem czasu Python stał się rozwijanym przez społeczność programistyczną projektem Open Source, nad którym czuwała organizacja non-profit założona przez Guido von Rossum'a, Python Software Foundation.

Python jest multi-paradygmatowym językiem, gdzie programowanie obiektowe i strukturalne są w pełni wspierane, wiele cech języka wspiera programowanie funkcyjne i aspektowe. Wiele innych paradygmatów jest wspieranych dzięki modularnym rozszerzeniom do języka.
Python w przeciwieństwie do języków statycznie typowanych takich jak C, Rust czy TypeScript, stosuje typowanie dynamiczne, które sprawdza poprawność i bezpieczeństwo typów w programie, dynamicznie, podczas jego egzekucji, w przeciwieństwie do typowania statycznego, gdzie typy sprawdzane są podczas kompilacji kodu. Dodatkowo Python posiada kombinacje zliczania referencji oraz cyklicznego Garbage Collector'a.
Architektura języka Python oferuje wsparcie, dla programowanie funkcyjnego zgodnego z tradycją języka Lisp. Język posiada typowe, dla języków funkcyjnych funkcje takie jak \lcode{filter}, \lcode{map}, \lcode{reduce}, \lcode{lambda}, list comprehensions, słowniki oraz generatory. Standardowa biblioteka języka posiada moduły \lcode{itertools} oraz \lcode{functools}, które implementują narzędzia funkcjonalne zapożyczone z języków Haskell oraz Standard ML.

Zamiast polegać na wbudowanej funkcjonalności w jądrze języka, Python został zaprojektowany tak, aby być najmożliwiej elastyczny, aby mógł współdziałać z różnymi odrębnymi modułami. Sama kompaktowa modularność sprawiła, że język stał się popularnym sposobem, na dodawanie programowalnego interfejsu to istniejących już aplikacji oraz języków programowania.
CPython jest referencyjną implementacją Python'a, napisaną w C, która spełnia standard C89 z niektórymi cechami standardu C99. Jedną z charakterystyk implementacji CPython'a jest to, że zespół odpowiedzialny za rozwój oraz utrzymanie kodu jego interpretera preferuje odrzucanie poprawek do kodu mających na celu marginalną poprawę szybkości i sprawności interpretera w zamian za zachowanie czystości i czytelności kodu źródłowego. Taka długoterminowa postawa umożliwiła powstanie powstanie innych implementacji języka Python opartych o inne rozwiązania, bądź inne języki za pomocą których napisany został sam interpreter. Dzięki różnorodności implementacji Python'a, jego modularności i łatwej rozbudowie o obce rozszerzenia programista może zdecydować się na przeniesienie wrażliwych na czas wykonywania funkcjonalności do odrębnych modułów napisanych w innych językach takich jak C, lub Rust, albo użyć wyspecjalizowanej do tego odmiany samego interpretera takiej jak PyPy, która posiada tak zwany JIT, czyli Just-In-Time compiler, pozwalający na optymalizację kodu na system bądź architekturę docelową danego programu.

Dzięki tak rozbudowanemu środowisku oraz społeczności otaczającej język powstała ogromna biblioteka paczek, dla języka Python, która często podawana za jedną z największych zalet języka. W ten sposób Python stał się swoistym odpowiednikiem \textit{Lingua Franca} wśród programistów. Z powodu wyżej wymienionych cech i okoliczności Python jest wykorzystywany w bardzo różniących się, oraz na pozór nie połączonych ze sobą dziedzinach takich jak Web Development, Automatyzacja, Bazy Danych, Aplikacje Mobilne, Testowanie Oprogramowania, Analiza Danych oraz Uczenie Maszynowe.



\section{Biblioteka PyTorch}
\label{section:pytorch}
PyTorch \cite{PyTorchWebSite} \cite{DeepLearningWithPyTorch} \cite{ProgrammingPyTorchForDeepLearning} \cite{PyTorchPocketReference} jest Open Source'ową biblioteką służącą do uczenia maszynowego. Jest bazowana na bibliotece Torch napisanej w języku Lua. Z powodu niszowości języka Lua, oraz braku modularności i możliwości rozbudowywania go o nowe funkcjonalności za pomocą zewnętrznych modułów i paczek, powstał PyTorch, czyli, biblioteka Torch, ale zaimplementowana w języku Python. Dzięki temu PyTorch może korzystać z bardzo rozbudowanego środowiska Python, które oferuje dużą ilość naukowych paczek, między innymi takich jak NumPy.


Jedną z największych zalet biblioteki PyTorch jest możliwość programowanie imperatywnego. Jest to przeciwieństwem do bibliotek takich jak TensorFlow i Keras, które ze względu na poleganie głównie na językach takich jak C i C++, oferują jedynie możliwość programowania symbolicznego. Większość Python'owego kodu jest imperatywne jako, że jest to dynamicznie interpretowany język. W sytuacji symbolicznej zachodzi przeciwieństwo, ponieważ istnieje bardzo wyraźne rozróżnienie pomiędzy zdefiniowaniem grafu komputacyjnego, a jego kompilacją. W przypadku imperatywnym komputacja zachodzi w momencie jej wywołania, nie we wcześniej zoptymalizowanym punkcie w kodzie. Podejście symboliczne pozwala na większą optymalizację, a imperatywne takie jak PyTorch pozwalaj na większą swobodność, oraz używanie natywnych cech, funkcjonalności i rozszerzających modułów języka Python.
Drugą największą zaletą biblioteki PyTorch są dynamiczne grafy komputacyjne, które w przeciwieństwie do bibliotek takich jak TensorFlow generują je statycznie przed uruchomieniem programu. PyTorch umie generować i modyfikować je dynamicznie podczas działania programu.

\clearpage

\section{Biblioteka NumPy}
\label{section:numpy}
NumPy \cite{PythonForDataAnalysis} \cite{SciPyAndNumPy} \cite{NumPyManual} \cite{WikipediaNumPy} jest Open Source biblioteką stworzoną, dla języka programowania Python, dodaje wsparcie, dla dużych wielowymiarowych tablic i macierzy wraz z dużą kolekcją wysokopoziomowych funkcji matematycznych, które pozwalają operować na wspomnianych macierzach. Poprzednikiem biblioteki NumPy był, Numeric, oryginalnie stworzony przez Jim'a Hugunin'a wraz z kontrybucjami kilku innych developerów. W roku 2005, Travis Oliphant stworzył projekt NumPy włączając w to właściwości oraz funkcjonalności Numeric'a.
Python nie był oryginalnie stworzony do numerycznej komputacji, ale już we wczesnym życiu języka różne towarzystwa naukowe i inżynieryjne wyrażały swoje zainteresowanie językiem.
NumPy adresuje problem powolności języka Python poprzez zapewnienie wielowymiarowych macierzy, funkcji i operacji, które są wydajne obliczeniowo operując na macierzach.

Używanie biblioteki NumPy w Python'ie funkcjonalnością przypomina programowanie w środowisku MATLAB, jako, że oba są interpretowane, mają podobną składnie, oba pozwalają użytkownikowi pisać szybkie i wydaje programy tak długo jak operacje przeprowadzane są na macierzach. W przeciwieństwie do MATLAB'a, NumPy nie oferuje tak wielkiej ilości dodatkowych narzędzi. Oferuje odwrotne podejście, gdzie to inne paczki/narzędzia korzystają u swoich podstaw z biblioteki NumPy. Dzięki zaawansowanej integracji z Python'em oraz byciu podłożem, dla zasadniczej większości paczek i narzędzi służących celom obliczeniowo naukowym takim jak SciPy, SymPy, Scikit-Learn i wielu innym NumPy, stał się podstawową warstwą, dla takich narzędzi a tym samym umożliwia im prostą komunikację między sobą jako, że macierze na których operują są macierzami biblioteki NumPy.

Główną funkcjonalnością biblioteki NumPy jest jej \lcode{ndarray}, struktura danych, reprezentująca \textit{n}-wymiarową macierz. Wewnętrzna niskopoziomowa implementacja owych macierzy polega na kroczących widokach w pamięci. Takowe dane muszą być homogenicznego typu. Takie widoki pamięci mogą być również bufferami zaalokowanymi z poziomu innych języków takich jak C, C++, czy Fortran. Powoduje to ogromną optymalizację, jako, że eliminuje to potrzebę kopiowania i przenoszenia danych.



\section{Biblioteka Matplotlib}
\label{section:matplotlib}

Biblioteka Matplotlib \cite{MatplotlibDocumentation} \cite{Matplotlib30Cookbook} \cite{PythonForDataAnalysis} \cite{WikipediaMatplotlib} jest najpopularniejszym Python'owym narzędziem służącym do tworzenia wykresów, grafów, histogramów, obrazów, wielowymiarowych grafów i rysunków służących do wizualizacji danych. Oryginalnie stworzona przez John D. Hunter'a miała za zadanie tworzyć przyzwoicie wyglądające wykresy, które można by poddać publikacji. Mimo, że inne biblioteki są dostępne większość programistów używa Matplotlib'a jako, że jest on najpopularniejszy oraz najbardziej i najlepiej rozbudowany ze wszystkich nie wspominając o jego wpasowaniu w istniejący ekosystem Data Science.

Matplotlib zapewnia obiektowo zorientowane API do tworzenia i używania generowanych wykresów w toolkitach GUI takich jak Tkinter wxPython, Qt, lub GTK. Dodatkowo istnieje również proceduralny interfejs ``pylab'' bazujący na maszynie stanu (podobnie do OpenGL), zaprojektowany, aby przypominać interface MATLAB'a.

Pyplot to moduł Matplotlib'a, który jest zaprojektowany z myślą bycia używalnym pod kątem programistycznym tak jak MATLAB, ale z dodatkową zaletą pozostawania w przestrzeni języka Python, która jest Open Source i darmowa.


\section{PyGame}
\label{section:pygame}

PyGame \cite{PyGameDocumentation} \cite{WikipediaPyGame} jest cross-platform'ową biblioteką stworzoną, dla języka Python zaprojektowaną z myślą tworzenia prostych gier i animacji. Zawiera w sobie biblioteki odpowiadające za grafikę i dźwięk.
PyGame, był oryginalnie napisany przez Pete Shinners'a, w celu zastąpienia PySDL po tym jak jego developemnt został zatrzymany. Od roku 2000 jest to projekt społeczności i został objęty licencją GNU Lesser General Public License, która pozwala na dystrybucje PyGame'a zarówno wraz z Open Source'owym oprogramowaniem jak i tym prawnie zastrzeżonym prawami autorskimi.




% \section{Klasy abstrakcyjne} ?????????
