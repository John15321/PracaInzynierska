% \section{Użyte narzędzia i technologie}



\section{Język Python}
Język \textbf{Python} jest bardzo popularnym, nowoczesnym oraz wysokopoziomowym językiem programowania. Czytając artykuły i inne treści na temat historii \textbf{Python} \cite{HistoriaPythona} \cite{WikiepdiaPythonProgrammingLanauge} dowiadujemy się, że język powstał w 1991 roku. Stworzony przez Guido van Rossum podczas swojej pracy w laboratorium w Centrum Matematyki i Informatyki w Amsterdamie, pierwotnie tworzony był z myślą zastąpienia rozwijanego w latach osiemdziesiątych języka \textbf{ABC}. Samą nazwę język Python nazwę zawdzięcza popularnemu serialowi komediowemu emitowanego przez BBC w latach siedemdziesiątych - ``Latający Cyrk Monty Pythona'', którego Guido był fanem.

Projekt początkowo zakładał stworzenie prostego w użyciu, zwięzłego i wysokopoziomowego języka, głównie na potrzeby pracy w Amsterdamskim laboratorium. Z biegiem czasu \textbf{Python} stał się rozwijanym przez społeczność projektem \textbf{Open Source}, nad którym czuwała organizacja non-profit założona przez Guido von Rossum'a, \textbf{Python Software Foundation}.

\textbf{Python} jest multi-paradygmatowym językiem, gdzie programowanie obiektowe i strukturalne są w pełni wspierane, wiele cech języka wspiera programowanie funkcyjne i aspektowe. Wiele innych paradygmatów jest wspieranych dzięki modularnym rozszerzeniom do języka.
\textbf{Python} w przeciwieństwie do języków statycznie typowanych takich jak \textbf{C}, \textbf{Rust} czy \textbf{TypeScript}, stosuje typowanie dynamiczne, które sprawdza poprawność i bezpieczeństwo typów w programie, dynamicznie, podczas jego egzekucji, w przeciwieństwie do typowania statycznego, gdzie typy sprawdzane są podczas kompilacji kodu. Dodatkowo \textbf{Python} posiada kombinacje zliczania referencji oraz cyklicznego garbage collector'a.




\section{Biblioteka PyTorch}


\section{Biblioteka NumPy}


\section{Biblioteka Matplotlib}


\section{Klasy abstrakcyjne}




\paragraph{Ukryty link do spisu treści}
Kliknij na numer strony, a pod nim masz ukryty odnośnik do spisu treści. Bardzo ułatwia skakanie po pdf'ie.

\paragraph{Używasz vscode?}
Zobacz na polecane wtyczki w README.

\paragraph{Jak wydzilić z tesktu}
Jak cokolwiek chcesz oddzielić z tekstu możesz użyć \lcode{onepage[Xmm]}.

\begin{onepage}[1cm]
    \centering
    Alias dolor odit. Velit et ut harum. Quos ullam enim suscipit qui omnis dolorum.
\end{onepage}

Reszta tekstu.
Doloremque voluptas sit mollitia eos ut aut. Qui et distinctio vitae. Possimus et in. Est provident qui sequi est nobis cupiditate magni. Recusandae animi aut non ea autem ipsam dolores hic repellendus.

I chyba nic więcej nie potrzeba. Powodzenia!
