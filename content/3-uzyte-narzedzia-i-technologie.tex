% \section{Użyte narzędzia i technologie}



\section{Język Python}
Język \textbf{Python} \cite{IntroducingPython} \cite{AdvancedPythonDevelopment} \cite{ExpertPythonProgramming} jest bardzo popularnym, nowoczesnym oraz wysokopoziomowym językiem programowania. Czytając artykuły i inne treści na temat historii \textbf{Python} \cite{HistoriaPythona} \cite{WikipediaPythonProgrammingLanauge} dowiadujemy się, że język powstał w 1991 roku. Stworzony przez Guido van Rossum podczas swojej pracy w laboratorium w Centrum Matematyki i Informatyki w Amsterdamie, pierwotnie tworzony był z myślą zastąpienia rozwijanego w latach osiemdziesiątych języka \textbf{ABC}. Samą nazwę język Python nazwę zawdzięcza popularnemu serialowi komediowemu emitowanego przez BBC w latach siedemdziesiątych - ``Latający Cyrk Monty Pythona'', którego Guido był fanem.
Projekt początkowo zakładał stworzenie prostego w użyciu, zwięzłego i wysokopoziomowego języka, głównie na potrzeby pracy w Amsterdamskim laboratorium. Z biegiem czasu \textbf{Python} stał się rozwijanym przez społeczność projektem \textbf{Open Source}, nad którym czuwała organizacja non-profit założona przez Guido von Rossum'a, \textbf{Python Software Foundation}.

\textbf{Python} jest multi-paradygmatowym językiem, gdzie programowanie obiektowe i strukturalne są w pełni wspierane, wiele cech języka wspiera programowanie funkcyjne i aspektowe. Wiele innych paradygmatów jest wspieranych dzięki modularnym rozszerzeniom do języka.
\textbf{Python} w przeciwieństwie do języków statycznie typowanych takich jak \textbf{C}, \textbf{Rust} czy \textbf{TypeScript}, stosuje typowanie dynamiczne, które sprawdza poprawność i bezpieczeństwo typów w programie, dynamicznie, podczas jego egzekucji, w przeciwieństwie do typowania statycznego, gdzie typy sprawdzane są podczas kompilacji kodu. Dodatkowo \textbf{Python} posiada kombinacje zliczania referencji oraz cyklicznego \textbf{Garbage Collector}'a.
Architektura języka \textbf{Python} oferuje wsparcie, dla programowanie funkcyjnego zgodnego z tradycją języka \textbf{List}. Język posiada typowe, dla języków funkcyjnych funkcje takie jak \lcode{filter}, \lcode{map}, \lcode{reduce}, \lcode{lambda}, list comprehensions, słowniki oraz generatory. Standardowa biblioteka języka posiada moduły \lcode{itertools} oraz \lcode{functools}, które implementują narzędzia funkcjonalne zapożyczone z języków \textbf{Haskell} oraz \textbf{Standard ML}.

Zamiast polegać na wbudowanej funkcjonalności w jądro języka, \textbf{Python} został zaprojektowany tak, aby być najmożliwiej elastyczny, aby mógł współdziałać z różnymi odrębnymi modułami. Sama kompaktowa modularność sprawiła, że język stał się popularnym sposobem, na dodawanie programowalnego interfejsu to istniejących już aplikacji oraz języków programowania.
\textbf{CPython} jest referencyjną implementacją \textbf{Python}'a, napisaną w \textbf{C}, która spełnia standard \textbf{C89} z niektórymi cechami standardu \textbf{C99}. Jedną z charakterystyk implementacji \textbf{CPython}'a jest to, że zespół odpowiedzialny za rozwój oraz utrzymanie kodu jego interpretera preferuje odrzucanie poprawek do kodu mających na celu marginalną poprawę szybkości i sprawności interpretera w zamian za zachowanie czystości i czytelności kodu źródłowego. Taka długoterminowa postawa umożliwiła powstanie powstanie innych implementacji języka \textbf{Python} opartych o inne rozwiązania, bądź inne języki za pomocą których napisany został sam interpreter. Dzięki różnorodności implementacji \textbf{Python}'a, jego modularności i łatwej rozbudowie o obce rozszerzenia programista może zdecydować się na przeniesienie wrażliwych na czas wykonywania funkcjonalności do odrębnych modułów napisanych w innych językach takich jak \textbf{C}, lub \textbf{Rust}, albo użyć wyspecjalizowanej do tego odmiany samego interpretera takiej jak \textbf{PyPy}, która posiada tak zwany \textbf{JIT}, czyli \textbf{Just-In-Time compiler}, pozwalający na optymalizację kodu na system bądź architekturę docelową danego programu.

Dzięki tak rozbudowanemu środowisku oraz społeczności otaczającej język powstała ogromna biblioteka paczek, dla języka \textbf{Python}, która często podawana za jedną z największych zalet języka. W ten sposób \textbf{Python} stał się swoistym odpowiednikiem \textit{Lingua Franca} wśród programistów. Przez co często dziedziny zajmujące się całkowicie różnymi dziedzinami takimi jak Web Development, Automatyzacja, Bazy Danych, Aplikacje Mobilne, Testowanie Oprogramowania, Analiza Danych oraz Uczenie Maszynowe.



\section{Biblioteka PyTorch}

\textbf{PyTorch} \cite{PyTorchWebSite} \cite{DeepLearningWithPyTorch} jest Open Source'ową biblioteką służącą do uczenia maszynowego. Jest bazowana na bibliotece \textbf{Torch} napisanej w języku \textbf{Lua}. Z powodu niszowości języka \textbf{Lua}, oraz braku modularności i możliwości rozbudowywania o nowe funkcjonalności za pomocą zewnętrznych modułów i paczek, powstał \textbf{PyTorch}, czyli, biblioteka \textbf{Torch}, ale zaimplementowana w języku \textbf{Python}. Dzięki temu \textbf{PyTorch} może korzystać z bardzo rozbudowanego środowiska \textbf{Python}, które oferuje dużą ilość naukowych paczek, między innymi takich jak \textbf{NumPy}.


Jedną z największych zalet biblioteki \textbf{PyTorch} jest możliwość programowanie imperatywnego. Jest to przeciwieństwem do bibliotek takich jak \textbf{TensorFlow} i \textbf{Keras}, które ze względu na poleganie głównie na językach takich jak \textbf{C} i \textbf{C++}, oferują jedynie możliwość programowania symbolicznego. Większość \textbf{Python}'owego kodu jest imperatywne jako, że jest to dynamicznie interpretowany język. W sytuacji symboliczne zachodzi przeciwieństwo, ponieważ zachodzi bardzo wyraźne rozróżnienie pomiędzy zdefiniowaniem grafu komputacyjnego, a jego kompilacją. W przypadku imperatywnym komputacja zachodzi w momencie jej wywołania, nie we wcześniej zoptymalizowanym punkcie w kodzie. Podejście symboliczne pozwala na większą optymalizację, a imperatywne takie jak \textbf{PyTorch} pozwalają na większą swobodność, oraz używanie natywnych cech, funkcjonalności i rozszerzających modułów języka \textbf{Python}.
Drugą największą zaletą biblioteki \textbf{PyTorch} są dynamiczne grafy komputacyjne, które w przeciwieństwie do bibliotek takich jak \textbf{TensorFlow} generują je statycznie przed uruchomieniem programu. \textbf{PyTorch} umie generować i modyfikować je dynamicznie podczas działania programu.

\clearpage

\section{Biblioteka NumPy}
\textbf{NumPy} \cite{PythonForDataAnalysis} \cite{SciPyAndNumPy} \cite{NumPyManual} \cite{WikipediaNumPy} jest Open Source biblioteką stworzoną, dla języka programowania \textbf{Python}, dodaje wsparcie, dla dużych wielowymiarowych tablic i macierzy wraz z dużą kolekcją wysokopoziomowych funkcji matematycznych, które pozwalają operować na wspomnianych macierzach. Poprzednikiem biblioteki \textbf{NumPy} był, \textbf{Numeric}, oryginalnie stworzony przez Jim'a Hugunin'a wraz z kontrybucjami kilku innych developerów. W roku 2005, Travis Oliphant stworzył projekt \textbf{NumPy} włączając w to właściwości oraz funkcjonalności \textbf{Numeric}'a.
\textbf{Python} nie był oryginalnie stworzony do numerycznej komputacji, ale już we wczesnym życiu języka różne towarzystwa naukowe i inżynieryjne wyrażały swoje zainteresowanie językiem.
\textbf{NumPy} adresuje problem powolności języka \textbf{Python} poprzez zapewnienie wielowymiarowych macierzy, funkcji i operacji, które są wydajne obliczeniowo operując na macierzach.

Używanie biblioteki \textbf{NumPy} w \textbf{Python}'ie funkcjonalnością przypomina programowanie w środowisku \textbf{MATLAB}, jako, że oba są interpretowane, mają podobną składnie, oba pozwalają użytkownikowi pisać szybkie i wydaje programy tak długo jak operacje przeprowadzane są na macierzach. W przeciwieństwie do \textbf{MATLAB}'a, \textbf{NumPy} nie oferuje tak wielkiej ilości dodatkowych narzędzi. Oferuje odwrotne podejście, gdzie to inne paczki/narzędzia korzystają u swoich podstaw z biblioteki \textbf{NumPy}. Dzięki zaawansowanej integracji z \textbf{Python}'em oraz byciu podłożem, dla zasadniczej większości paczek i narzędzi służących celom obliczeniowo naukowym takim jak \textbf{SciPy}, \textbf{SymPy}, \textbf{Scikit-Learn} i wielu innym \textbf{NumPy} stał się podstawową warstwą, dla takich narzędzi a tym samym umożliwia im prostą komunikację między sobą jako, że macierze na których operują są macierzami biblioteki \textbf{NumPy}.

Główną funkcjonalnością biblioteki \textbf{NumPy} jest jej \lcode{ndarray}, struktura danych, reprezentująca \textit{n}-wymiarową macierz. Wewnętrzna niskopoziomowa implementacja owych macierzy polega na kroczących widokach w pamięci. Takowe dane muszą być homogenicznego typu. Takie widoki pamięci mogą być również bufferami zaalokowanymi z poziomu innych języków takich jak \textbf{C}, \textbf{C++}, czy \textbf{Fortran}. Powoduje to ogromną optymalizację, jako, że eliminuje to potrzebę kopiowania i przenoszenia danych.



\section{Biblioteka Matplotlib}

Biblioteka \textbf{Matplotlib} \cite{MatplotlibDocumentation} \cite{Matplotlib30Cookbook} \cite{PythonForDataAnalysis} \cite{WikipediaMatplotlib} jest najpopularniejszym \textbf{Python}'owym narzędziem służącym do tworzenia wykresów, grafów, histogramów, obrazów, wielowymiarowych grafów i rysunków służących do wizualizacji danych. Oryginalnie stworzona prze z John D. Hunter'a miała za zadanie tworzyć przyzwoicie wyglądające wykresy, które można by poddać publikacji. Mimo, że inne biblioteki są dostępne większość programistów używa \textbf{Matplotlib}'a jako, że jest on najpopularniejszy oraz najbardziej i najlepiej rozbudowany ze wszystkich nie wspominając o jego wpasowaniu w istniejący ekosystem Data Science.

\textbf{Matplotlib} zapewnia obiektowo zorientowane \textbf{API} do tworzenia i używania generowanych wykresów w toolkitach \textbf{GUI} takich jak \textbf{Tkinter} \textbf{wxPython}, \textbf{Qt}, lub \textbf{GTK}. Dodatkowo istnieje również proceduralny interfejs ``\textbf{pylab}'' bazujący na maszynie stanu (podobnie do \textbf{OpenGL}), zaprojektowany, aby przypominać interface \textbf{MATLAB}'a.

\textbf{Pyplot} to moduł \textbf{Matplotlib}'a, który jest zaprojektowany z myślą bycia używalnym pod kątem programistycznym tak jak \textbf{MATLAB}, ale z dodatkową zaletą pozostawania w przestrzeni języka \textbf{Python}, która jest Open Source i darmowa.


\section{PyGame}

\textbf{PyGame} \cite{PyGameDocumentation} \cite{WikipediaPyGame} jest cross-platform'ową biblioteką stworzoną, dla języka \textbf{Python} zaprojektowaną z myślą tworzenia prostych gier i animacji. Zawiera w sobie biblioteki odpowiadające za grafikę i dźwięk.
\textbf{PyGame}, był oryginalnie napisany przez Pete Shinners'a, w celu zastąpienia \textbf{PySDL} po tym jak jego developemnt został zatrzymany. Od roku 2000 jest to projekt społeczności i został objęty licencją \textbf{GNU Lesser General Public License}, która pozwala na dystrybucje \textbf{PyGame}'a zarówno wraz z Open Source'owym oprogramowaniem jak i tym prawnie zastrzeżonym prawami autorskimi.




% \section{Klasy abstrakcyjne} ?????????
