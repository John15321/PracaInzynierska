


\section{Gra Snake}

Snake \cite{SnakeGame} jest popularnym konceptem prostej gry komputerowej, gdzie gracz ma za zadanie manewrować poruszającą się linią w taki sposób, aby nie kolidowała ze sobą lub, dodatkowo ściankami planszy, w między czasie zbierając punkty losowo porozmieszczane po mapie. Linia ma za zadanie reprezentować ciało węża, który pełza, a punkty reprezentowane są w postaci owocu, lub prostej grupki pikseli.
Odmiana użyta w celu testowania sieci neuronowej została na pisana w języku Python, przy użyciu paczki PyGame, która posłuży do reprezentacji graficznej postępowania sztucznej inteligencji, oraz samej rozgrywki.
Wąż został przestawiony w postaci linii składającej się z pojedynczych zielonych kwadratów. Wąż zaczyna z trzema zielonymi kwadratami, każdy dodatkowo zdobyty punkt podłuża węża o jeden kwadrat. Punkty możliwe do zdobycia są reprezentowany w postaci czerwonych kwadratów.

\begin{figure}[h!]
    \centering
    \mediumimage{img/snake1.png}
    \caption{Gra Snake}
    \label{img:snake_game}
\end{figure}


\section{Implementacja gry}


Na początku zaimportowane zostają moduły potrzebne do uruchomienia gry i do jej wewnętrznych procesów:


\begin{onepage}
    \begin{lstlisting}[
        caption={},
        label=code:,
        language={Python},
        numbers=none,
        ]
        import random
        from enum import Enum
        from typing import Tuple
        
        import numpy as np
        import pygame
    \end{lstlisting}
\end{onepage}



Stworzenie pomocniczych klas służących jako enumeratory na kolory i kierunki:


\begin{onepage}
    \begin{lstlisting}[
        caption={},
        label=code:,
        language={Python},
        numbers=none,
        ]
        class GameColors(Enum):
            WHITE: Tuple = (255, 255, 255)
            RED: Tuple = (150, 0, 0)
            DARK_GREEN: Tuple = (0, 60, 10)
            LIGHT_GREEN: Tuple = (50, 160, 80)
            BLACK: Tuple = (0, 0, 0)
        
        
        class GameDirection(Enum):
            RIGHT = 1
            LEFT = 2
            UP = 3
            DOWN = 4
        
    \end{lstlisting}
\end{onepage}

\clearpage

Klasa tworząca abstrakcję punktu na mapie powstałego z pikseli:

\begin{onepage}
    \begin{lstlisting}[
        caption={},
        label=code:,
        language={Python},
        numbers=none,
        ]
    class GamePointOnTheMap:
        def __init__(self, x: int, y: int) -> None:
            self._x = None
            self._y = None
    
            self.x = x
            self.y = y
    
        def __eq__(self, other):
            if self.x == other.x and self.y == other.y:
                return True
            return False
    
        @property
        def x(self):
            return self._x
    
        @property
        def y(self):
            return self._y
    
        @x.setter
        def x(self, x):
            self._x = x
    
        @y.setter
        def y(self, y):
            self._y = y
    \end{lstlisting}
\end{onepage}

Definicja głównej klasy odpowiedzialnej za mechanikę gry. W jej konstruktorze zainicjalizowane zostaje okno graficzne służące wizualizacji gry:

\begin{onepage}
    \begin{lstlisting}[
        caption={},
        label=code:,
        language={Python},
        numbers=none,
        ]
    class ExampleSnakeGame:
        def __init__(self, w: int = 640, h: int = 480):
            self.w = w
            self.h = h
            self.display = pygame.display.set_mode((self.w, self.h))
            pygame.display.set_caption("Snake")
            self.clock = pygame.time.Clock()
            self.reset_game_state()
    
    \end{lstlisting}
\end{onepage}

\clearpage

Funkcja odpowiadająca za resetowanie gry:

\begin{onepage}
    \begin{lstlisting}[
        caption={},
        label=code:,
        language={Python},
        numbers=none,
        ]

    def reset_game_state(self):
        # Initialize game state
        self.direction = GameDirection.RIGHT

        self.head = GamePointOnTheMap(self.w / 2, self.h / 2)
        self.snake = [
            self.head,
            GamePointOnTheMap(self.head.x - BLOCK_SIZE, self.head.y),
            GamePointOnTheMap(self.head.x - (2 * BLOCK_SIZE), self.head.y),
        ]

        self.score = 0
        self.food = None
        self._place_food()
        self.frame_iteration_number = 0

    \end{lstlisting}
\end{onepage}


Prywatna funkcja odpowiadająca za losowanie miejsca pojawienia się jedzenia, dla węża na mapie:

\begin{onepage}
    \begin{lstlisting}[
        caption={},
        label=code:,
        language={Python},
        numbers=none,
        ]
    def _place_food(self):
        x = random.randint(0, (self.w - BLOCK_SIZE) // BLOCK_SIZE) * BLOCK_SIZE
        y = random.randint(0, (self.h - BLOCK_SIZE) // BLOCK_SIZE) * BLOCK_SIZE
        self.food = GamePointOnTheMap(x, y)
        if self.food in self.snake:
            self._place_food()
    \end{lstlisting}
\end{onepage}

\clearpage

Funkcja \lcode{play_step} jest odpowiedzialna za każdą klatkę rozgrywki, na początku sprawdza czy użytkownik sie chce wyjść z rozgrywki, następnie wykonuje ruchy węża jakie miały nastąpić. Na podstawie wykonanego ruchu ocenia styację rozgrywki, i decyduje o tym jaka powinna być nagroda za dany ruch. Gdzie zdobycie jedzenia to dodatkowe 10 punktów, nie zdobycie jedzenia to 0 punktów, a śmierć ujemne 10 punktów. Następnie zakładając, że wąż żyje, jedzenie jest umieszczane w innym miejscu, lub wąż wykonuje ruch. Po czym następuje aktualizacja interfejsu o kolejną klatkę i zwrócenie informacji o danej klatce takich jak nagroda i stan gry.

\begin{onepage}
    \begin{lstlisting}[
        caption={},
        label=code:,
        language={Python},
        numbers=none,
        ]

    def play_step(self, action) -> Tuple[int, bool, int]:
        self.frame_iteration_number += 1
        # 1. collect user input
        for event in pygame.event.get():
            if event.type == pygame.QUIT:
                pygame.quit()
                quit()
        # 2. move
        self._move(action)  # update the head
        self.snake.insert(0, self.head)
        # 3. Update the game state
        # Eat food: +10
        # Game Over: -10
        # Else: 0
        reward = 0
        game_over = False
        if self.is_collision() or self.frame_iteration_number>100*len(self.snake):
            game_over = True
            reward = -10
            return reward, game_over, self.score
        # 4. place new food or just move
        if self.head == self.food:
            self.score += 1
            reward = 10
            self._place_food()
        else:
            self.snake.pop()
        # 5. update ui and clock
        self._update_ui()
        self.clock.tick(SPEED)
        # 6. return game over and score
        return reward, game_over, self.score
    \end{lstlisting}
\end{onepage}

\clearpage

Pomocnicza funkcja, której zadaniem jest wykrycie kolizji obiektów:


\begin{onepage}
    \begin{lstlisting}[
        caption={},
        label=code:,
        language={Python},
        numbers=none,
        ]

    def is_collision(self, point=None):
        if point is None:
            point = self.head
        # Check if it hits the boundary
        if (
            point.x > self.w - BLOCK_SIZE
            or point.x < 0
            or point.y > self.h - BLOCK_SIZE
            or point.y < 0
        ):
            return True
        # Check if it hits itself
        if point in self.snake[1:]:
            return True
        return False
    \end{lstlisting}
\end{onepage}

Pomocnicza prywatna funkcja odpowiedzialna za aktualizację interfejsu graficznego:

\begin{onepage}
    \begin{lstlisting}[
        caption={},
        label=code:,
        language={Python},
        numbers=none,
        ]

    def _update_ui(self):
        self.display.fill(GameColors.BLACK.value)
        for pt in self.snake:
            pygame.draw.rect(
                self.display,
                GameColors.DARK_GREEN.value,
                pygame.Rect(pt.x, pt.y, BLOCK_SIZE, BLOCK_SIZE),
            )
            pygame.draw.rect(
                self.display,
                GameColors.LIGHT_GREEN.value,
                pygame.Rect(pt.x + 4, pt.y + 4, 12, 12),
            )
        pygame.draw.rect(
            self.display,
            GameColors.RED.value,
            pygame.Rect(self.food.x, self.food.y, BLOCK_SIZE, BLOCK_SIZE),
        )
        text = font.render(f"Score: {self.score}", True, GameColors.WHITE.value)
        self.display.blit(text, [0, 0])
        pygame.display.flip()
    \end{lstlisting}
\end{onepage}

\clearpage

Pomocnicza prywatna funkcja odpowiadająca za obliczanie następnej pozycji węża:

\begin{onepage}
    \begin{lstlisting}[
        caption={},
        label=code:,
        language={Python},
        numbers=none,
        ]

    def _move(self, action):
        # [straight, right, left]
        clock_wise = [
            GameDirection.RIGHT,
            GameDirection.DOWN,
            GameDirection.LEFT,
            GameDirection.UP,
        ]
        move_index = clock_wise.index(self.direction)
        if np.array_equal(action, [1, 0, 0]):
            new_direction = clock_wise[move_index]  # no change
        elif np.array_equal(action, [0, 1, 0]):
            next_idx = (move_index + 1) % 4
            new_direction = clock_wise[next_idx]
        else:  # [0, 0, 1]
            next_idx = (move_index - 1) % 4
            new_direction = clock_wise[next_idx]
        self.direction = new_direction
        x = self.head.x
        y = self.head.y
        if self.direction == GameDirection.RIGHT:
            x += BLOCK_SIZE
        elif self.direction == GameDirection.LEFT:
            x -= BLOCK_SIZE
        elif self.direction == GameDirection.DOWN:
            y += BLOCK_SIZE
        elif self.direction == GameDirection.UP:
            y -= BLOCK_SIZE
        self.head = GamePointOnTheMap(x=x, y=y)
    \end{lstlisting}
\end{onepage}



\clearpage

\section{Przykład użycia}

Większość mechaniki gry została zaimplementowana w klasie \lcode{ExampleSnakeGame}, która odpowiada za mechanikę gry oraz przechowywanie jej stanu. Znajdują się w niej między innymi mechaniki odpowiedzialne za wykrywanie zderzeń, poruszanie się węża, sterowanie wężem, czy losowanie miejsca pojawienia się następnych punktów. Dodatkowo zaimplementowane zostały klasy enumeratorowe, takie jak \lcode{GameDirection} i \lcode{GamePointOnTheMap} w celu zwiększenia czytelności kodu:

\begin{onepage}
    \begin{lstlisting}[
        caption={Importowanie gry},
        label=code:import_snake,
        language={Python},
        numbers=none,
        ]
        import numpy as np
        import torch
        from sgai.agent.trainer import Trainer
        
        from snake import GameDirection, GamePointOnTheMap, ExampleSnakeGame
    \end{lstlisting}
\end{onepage}



Następnie definiujemy potrzebny interfejs \lcode{Trainer}, aby paczka \lcode{sgai} umiała się komunikować z grą. Zaczynami od definicji klasy i jej konstruktora. W tym przykładzie wielkość warstwy ukrytej pozostawiamy nie zmienioną ponieważ podstawowa w pełni wystarczy do poziomu skomplikowania przygotowanej gry:

\begin{onepage}
    \begin{lstlisting}[
        caption={Definicja klasy \lcode{MyTrainer}},
    label=code:definition_my_trainer,
    language={Python},
    numbers=none
]
class MyTrainer(Trainer):
    def __init__(self, game, input_size: int, output_size: int):
        super().__init__(game, input_size, output_size)
    \end{lstlisting}
\end{onepage}

Następnie definiujemy dwie funkcje \lcode{get_state()}, oraz \lcode{perform_action()}. Są one interfejsem pomiędzy paczką \lcode{sgai}, a kodem gry, dlatego to użytkownik paczki musi je zaimplementować, ponieważ paczka nie ma możliwości posiadania wiedzy na temat własności działania kodu gry na, której będzie szkolić sieć neuronową. Na samym początku definiujemy stan gry, pobierając konkretne elementy rozgrywki takie jak pozycja węża, pozycja nagrody itp.:

\begin{onepage}
    \begin{lstlisting}[
        caption={Definicja funkcji \lcode{get_state()}},
    label=code:definition_get_state,
    language={Python},
    numbers=none
]
    def get_state(self) -> np.ndarray:
        head = self.game.snake[0]
        point_left = GamePointOnTheMap(head.x - 20, head.y)
        point_right = GamePointOnTheMap(head.x + 20, head.y)
        point_up = GamePointOnTheMap(head.x, head.y - 20)
        point_down = GamePointOnTheMap(head.x, head.y + 20)

        left_direction = self.game.direction == GameDirection.LEFT
        right_direction = self.game.direction == GameDirection.RIGHT
        up_direction = self.game.direction == GameDirection.UP
        down_direction = self.game.direction == GameDirection.DOWN
    \end{lstlisting}
\end{onepage}

W tym przykładzie zebranych będzie łącznie $11$ parametrów opisujących stan gry, które następnie zostaną przekazane sieci neuronowej jako wejście. Gdzie każda z wartości może być jedynie $0$, lub $1$, $0$ oznaczające \lcode{False}, a $1$ \lcode{True}:

\begin{itemize}
    \item Czy jakieś niebezpieczeństwo znajduje się naprzeciwko?
    \item Czy niebezpieczeństwo znajduje się po prawej?
    \item Czy niebezpieczeństwo znajduje się po lewej?
    \item Czy wąż porusza się w lewo?
    \item Czy wąż porusza się w prawo?
    \item Czy wąż porusza się w górę?
    \item Czy wąż porusza się w dół?
    \item Czy jedzenie znajduje się po lewej?
    \item Czy jedzenie znajduje się po prawej?
    \item Czy jedzenie znajduje się nad wężem?
    \item Czy jedzenie znajduje się pod wężem?
\end{itemize}


\begin{onepage}
    \begin{lstlisting}[
    caption={Zwracany stan},
    label=code:return_state,
    language={Python},
    numbers=none
]
state = [
    # Danger is straight ahead
    (right_direction and self.game.is_collision(point_right))
    or (left_direction and self.game.is_collision(point_left))
    or (up_direction and self.game.is_collision(point_up))
    or (down_direction and self.game.is_collision(point_down)),
    # Danger is on the right
    (up_direction and self.game.is_collision(point_right))
    or (down_direction and self.game.is_collision(point_left))
    or (left_direction and self.game.is_collision(point_up))
    or (right_direction and self.game.is_collision(point_down)),
    # Danger is on the left
    (down_direction and self.game.is_collision(point_right))
    or (up_direction and self.game.is_collision(point_left))
    or (right_direction and self.game.is_collision(point_up))
    or (left_direction and self.game.is_collision(point_down)),
    # Current move direction
    left_direction,
    right_direction,
    up_direction,
    down_direction,
    # Food location
    self.game.food.x < self.game.head.x,  # Food is on the left
    self.game.food.x > self.game.head.x,  # Food is on the right
    self.game.food.y < self.game.head.y,  # Food is up
    self.game.food.y > self.game.head.y,  # Food is down
]

return np.array(state, dtype=int)
    \end{lstlisting}
\end{onepage}

\clearpage

Ponieważ wewnętrzna funkcja \lcode{play_step()} gry sama zwraca nagrodę, stan oraz wynik gry (które są wynikiem podjętej przez nas akcji), zaimplementowanie funkcji \lcode{perform_action()} jest bardzo proste:


\begin{onepage}
    \begin{lstlisting}[
    caption={Implementacja funkcji \lcode{perform_action()}},
    label=code:definition_perform_action,
    language={Python},
    numbers=none
]
    def perform_action(self, final_move) -> Tuple[int, bool, int]:
        reward, game_over, score = self.game.play_step(final_move)
        return reward, game_over, score
    \end{lstlisting}
\end{onepage}


Tworzymy obiekt klasy \lcode{MyTrainer}, definiujemy jego parametry i uruchamiamy trening sieci neuronowej:

\begin{onepage}
    \begin{lstlisting}[
    caption={},
    label=code:,
    language={Python},
    numbers=none
]
if __name__ == "__main__":
    mt = MyTrainer(game=ExampleSnakeGame(), input_size=11, output_size=3)
    mt.train(model_file="model.pth")

\end{lstlisting}
\end{onepage}

Jak widać na poniższym wykresie w okolicach setnej rozgrywki, sieć neuronowa znacznie się poprawia. Następnie przez kolejne kilkaset gier widoczny jest jedynie marginalny postęp, przez to można stwierdzić że nastąpiło nasycenie spowodowane prostą budową sieci.

\begin{figure}[h]
    \centering
    \bigimage{img/ai_graph2.png}
    \caption{Wykres nauki sieci neuronowej}
    \label{img:ai_graph}
\end{figure}
